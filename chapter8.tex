\chapter{Summary and future work}
\label{chap:conclusions}
\chaptermark{Summary and future work}

Grid frequency regulation will be an increasing challenge as wind energy and other renewable energy sources continue to grow. Developing controllers that allow wind farms to provide secondary frequency regulation for the grid effectively will therefore speed adoption of renewable energy sources in the future. This thesis works towards an integrated closed-loop model-based controller by deriving wake models for control applications, implementing sensing and estimation methods, and developing model-based receding horizon controllers.

While wake models have been extensively developing for decades, these models generally focused on maximizing power or minimizing loads in steady-state. In the context of frequency regulation, however, the dynamics of wind farm wake interactions become increasingly important, and a dynamic wake model is needed to develop control algorithms. As importantly, a control oriented dynamic wake model must be computationally efficient, solvable in fractions of a second. In Chapter~\ref{chap:dynwake} we develop a dynamic model of a wind farm wake from first principles. By making a few simplifying assumptions, such as linearized advection and time-invariance of the mixing rate of the wake with the surrounding air, the model reduces to a one-dimensional partial differential equation for the wake velocity deficit. In Section~\ref{sec:dynwake-1d} the model is further simplified by considering regular wind farm arrangements and using the Jensen model's square superposition approach. When compared to numerical simulations of a wind farm at start up, this simple dynamic wake model can predict the velocity at downstream wind turbines with enough accuracy to be included in a model-based control algorithm with closed-loop feedback.

Noticing the considerable disagreement of models of yawed turbines, a lifting line theory for the inviscid flow past a yawed wind turbine is developed in Chapter~\ref{chap:yaw}. Drawing on an analogy between a finite airfoil, the transverse lift distribution of a yawed actuator disk is shown to be elliptic. This elliptic distribution induces a constant transverse velocity behind the turbine, which can be combined with axial momentum conservation to derive expressions for the streamwise velocity deficit behind the turbine and the disk-averaged velocity through the rotor. When compared to simulations of yawed actuator disks, this lifting line theory more closely agrees than previous models. Finally, the lifting line theory is used as an initial condition for the wake model of Chapter~\ref{chap:dynwake}, matching measurements from the experiments of Bastankhah and Port\'{e}-Agel~\cite{Bastankhah2016a}. 

Although the dynamic model in Section~\ref{sec:dynwake-1d} provides acceptable agreement with numerical simulations, power measurements from wind farms can be used to improve the estimation capabilities of the model. The dynamic wake model, which is a nonlinear PDE, cannot be used directly in standard Kalman filter methods, however, because of the large state space and nonlinearity. Instead, we use an ensemble-based method that converges to the canonical Kalman filter and can be implemented in a computationally efficient way. Using an ensemble Kalman filter, we show in Chapter~\ref{chap:estimation} that power measurements can be used to improve estimates of the wake velocity deficit, wake expansion coefficients, and the inflow velocity. This approach is implemented in LES, providing good agreement with measured velocities and best-fit wake expansion coefficients.

In Chapter~\ref{chap:rhc} we build a model-based receding horizon controller around the wake model of Section~\ref{sec:dynwake-1d} and the state and parameter estimation of Chapter~\ref{chap:estimation}. This method is implemented using the PDE-constrained optimal control theories of Section~\ref{sec:methods-pdeopt}, and can therefore be applied in real time. Using LES as a ``virtual wind farm" we show that controlled wind farm can successfully track power reference signals and passes many of the qualification tests used by PJM. When compared against a static wake model-based controller, the dynamics of the wake model is shown to be important as the controlled wind farm of the static model cannot follow power reference signals. In Chapter~\ref{chap:rhc} we move toward pitch and generator torque based controls and demonstrate the potential improvements gained by this adaptation.

This study demonstrates that model-based receding horizon control can be used to effectively provide secondary frequency regulation services using wind farms. The results demonstrate that for the test system used (LES with the actuator disk turbine model) and the given initial conditions, up-regulation can be provided effectively with derates that are less than the maximum up-regulation requested. Previous studies~\cite{Aho2013a, Aho2014a, Jeong2014a} required a derate equal to the full magnitude of the largest increased power of the reference signal. The potential for reducing the required derate has important economic implications for wind farm operators. This derate reduction would mean that wind farms could generate more revenue from bulk power supply while still providing the same level of secondary frequency regulation. As a result, the cost-effectiveness of secondary regulation with wind farms could be altered significantly and wind farms may become a low-cost and dependable provider of secondary frequency regulation.

\section{Directions for future work}
While this thesis presents significant progress in allowing wind farms to provide secondary frequency regulation, more work is needed to fully implement the control approach presented in this thesis. The control algorithm presented in Chapter~\ref{chap:rhc2} needs to be tested in a high-fidelity simulation environment. Further tests in actuator line simulations or field tests would also be needed to fully vet the control algorithms. We also presented a yaw model that addresses limitations in the existing literature, but more work is needed implement and test the model in the context of wind farms with multiple turbines, implement sensing and estimation methods, and build a model-based controller for secondary frequency regulation. Finally, the approach of Chapter~\ref{chap:rhc} and~\ref{chap:rhc2} assumed that the regulation signal is known in advance instead of in real time. In order to implement the control in a realistic setting, forecasts of future regulation signals or control designs that maximize flexibility are needed.

Fortunately, the research presented in this thesis is part of a wide ranging effort from multiple research groups to advance wind energy integration into the power grid. No doubt significant advances will be made in the coming years. In the realm of yawed wind turbine research, researchers are rapidly understanding the wake behind a yawed turbine. Since initial observations of the curled wake several years ago~\cite{Howland2016a}, recent studies have shown the effect of wind shear, turbulence, and atmospheric stability on the evolution of the curled wake~\cite{Bartl2018a, Vollmer2016a}. The results of Chapter~\ref{chap:yaw} have already been used as an initial condition in an eddy-viscosity type model~\cite{Martinez2018a}. This rapid development will further our ability to model wind farms and improve control designs.

Recent work has also tried to address the down regulation strategy trade-offs alluded to in Chapter~\ref{chap:rhc2}. Specifically, what is the best downrate strategy to store kinetic energy in the rotor and reduce the thrust coefficient to maximize up-regulation potential? The approach of Chapter~\ref{chap:rhc2} attempts to computationally answer this equation, but other recent research~\cite{Hoek2018a, Kazda2018a, Siniscalchi-Minna2018a} is also trying to advance understanding by using power and thrust coefficient curves and quantifying power availability. In the coming years, a better understanding of the advantages of and potential trade-offs between storing energy in the rotation of the rotor and the flow field will improve control designs.

Since the beginning of this doctoral research, several new dynamic modeling and control approaches have been developed. PI control for secondary frequency regulation is still being studied and developed~\cite{vanWingerden2017a}. Dynamic yaw and thrust modulation has been used for power maximization~\cite{Munters2018a}. New controls-oriented models, such as 2D Navier-Stokes~\cite{Boersma2018a} and data-driven models~\cite{King2018a, Adcock2018a}, show promise. Low-resolution LES with adjoint-based estimation is an emerging research area~\cite{Bauweraerts2018a}. As these control methods mature and develop, a comprehensive comparison of their strengths and weaknesses will be needed to drive innovation.

The research presented in this thesis is ultimately part of a vibrant community of wind energy researchers trying to push our understanding of wind farm aerodynamics and improve control algorithms. As this research continues, our ability to increase the penetration of wind energy on the electric power system will continue to expand. With continued work, we can help meet the world's commitment to renewable energy and start to address the enormous challenges posed by global climate change.






